\chapter{Problem Description}
\label{cha:problem_description}

    This report describes our model of the "rotary flexible joint" device made by Quanser and the control schemes we designed to control it.

    The system is composed by a DC motor that provides torque to a metal base, over which a metal arm is fixed with an hinge and two symmetrical springs.

    The length of the arm, hence its inertia, and the equilibrium length of the springs can be modified in a variety of different configurations.

    \image{./images/Chapter 1/system.png}
    The system has several interfaces that could be connected to an acquisition system (ADC/DAC + Amplifier) to acquire measurements and provide input signal, namely:
    \begin{itemize}
        \item Actuators:
            \subitem Voltage driven DC motor;
        \item Sensors:
            \subitem Incremental Encoder for the position of the base \wrt the global reference frame;
            \subitem Incremental Encoder for the relative position of the arm \wrt the base.
    \end{itemize}
    The acquisition system, composed of ADC/DAC + Amplifier model, is out of the scope of this report.

    In the tables below we can see respectively the datasheet for the motor and for the flexible joint:
    \image{./images/Chapter 1/Motor data.png}

    \image{./images/Chapter 1/Flexible joint data.png}

    The main task of this project is to provide a basic control strategy for such system and to develop a more advanced control strategy to accommodate all possible configurations of the system in question.
    
    This goal is divided in sub-tasks to be achieved:
    \begin{enumerate}
        \item Position control of the base, with a frequency based approach;
        \item Position control of the arm tip, with a frequency based and a state space approaches;
        \item Manage uncertainties and control the position of the arm tip with the system in different configurations, with a state space approach and advanced control techniques (robust or adaptive).
    \end{enumerate}





