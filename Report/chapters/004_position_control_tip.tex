\chapter{Position Control of The Tip}
\label{cha:position_tip}

% \section{Research of the machine paramethers}
% \label{sec:reserche_machine_paramethers}
% The DTC method needs only a few paramethers due to the fact that the current saturation is already handled in the computation of the maximum torque and most of the control variables are tuned empirically.
% In fact the 3 needed paramethers are:
% \begin{itemize}
%     \item rated speed: 
%             \begin{equation}
%                 \omega_{m_{rated}} = \frac{2\pi f}{n_p} = 104.72 rad/s
%                 \label{eq:omega_m_rated}
%             \end{equation}

%     \item maximum torque:
%             \begin{equation}
%                 me_{max} = \frac{V_{rated}^2}{((\frac{3}{2} (2 \pi f)^2 L_s ((\frac{L_s L_r}{L_m^2})-1)))} = 203.99 Nm
%                 \label{eq:me_max}
%             \end{equation}

%     \item reference stator flux(to use the ferromagnetic material at its maximum):
%             \begin{equation}
%                 \psi_{s_{ref}} = \frac{V_{rated} \sqrt{3}}{2 \pi f} = 2.095 Wb
%                 \label{eq:psi_s_ref}
%             \end{equation}

% \end{itemize}
% \newpage

% \section{Development of the Machine Model}
% \label{sec:machine_model}
% The following scheme in figure \ref*{fig: machine_model} represents the machine model.
% The block scheme uses blocks of Simulink to describe the entire behavior of the motor using differential equations.
% \begin{figure}[h!]
%     \centering
%     \includegraphics[width=1\textwidth]{machine_model.png}
%     \caption{Model of the machine in Simulink}
%     \label{fig: machine_model}
% \end{figure}

% The next ones are all the components used to control the control variables, the 3-phase voltage $V_s$ indeed, and estimate all the control references using the 3-phase vector of the stator current $i_s$.
% Inside the model there are some Complex Integrator that allow to integrete the real part and the imaginary parts of the state vectors separatly and after recompose the originally space vectors.
% \begin{figure}[h!]
%     \centering
%     \includegraphics[width=1\textwidth]{complex_integrator.png}
%     \caption{Scheme of a Complex integrator}
%     \label{fig: complex_integrator}
% \end{figure}
% \newpage

% \section{Development of the Control Unit}
% \label{sec: control_unit}
% The following scheme repreesents the implementation of the controller with the DTC method. This method is based on the dynamic approximation $V_s = \Delta\psi$ that can applies only for high power machine keeping negligible the voltage dissipated on the $R_s$. 
% \begin{figure}[h!]
%     \centering
%     \includegraphics[width=1\textwidth]{DTC_hexagon.png}
%     \caption{Scheme of the stator phase control}
%     \label{fig: DTC_hexagon}
% \end{figure}
% The idea is to keep bounded the magnitude of the $\psi_s$ continuosly switching between different configuation, depending on the sector in which the $\psi_s$ is, (according to $\varphi$) of the 3-phase $V_s$ accorting with the switching table in figure \ref*{fig: switching_table} to keep in rotation the the flux.
% In order to control instead the speed (unless the motor will rotate at the maximum speed) you control the torque, which is controlled by appling null configuaration of the $V_s$ (according to $\tau$) accorting with the switching table in figure \ref*{fig: switching_table}.
% \begin{figure}[h!]
%     \centering
%     \includegraphics[width=0.75\textwidth]{switchin_table.png}
%     \caption{Switching table used in DTC}
%     \label{fig: switching_table}
% \end{figure}
% \\
% The implementation is made by sending to a block that manages the switchin table the two signal $\varphi$ and $\tau$ computed using 2 hysteresis controllers as represented in figure \ref*{fig: hysteresis_controllers}.
% \begin{figure}[h!]
%     \centering
%     \includegraphics[width=0.75\textwidth]{hysteresis_controllers.png}
%     \caption{The hysteresis controllers}
%     \label{fig: hysteresis_controllers}
% \end{figure}
% \\
% In the Simulink implementation is similar to the block scheme, the only differences are:
% \begin{itemize}
%     \item The two side hysteresis for the torque is made with two hysteresis blocks;
%     \item The sector recognition is done evaluating the phase shift of the $\psi_s$;
%     \item The magnitude of the stator flux is evalated comparing the estimated magnitude with a reference value.
% \end{itemize}
% The outputs from these comparisons are the input of a block that implements the switching table as reported in figure \ref*{fig: DTC_controller}.
% \begin{figure}[h!]
%     \centering
%     \includegraphics[width=1\textwidth]{DTC_control_scheme.png}
%     \caption{The DTC control scheme created in Simulink}
%     \label{fig: DTC_controller}
% \end{figure}
% \\
% The inputs of this controller is $m_{e_{err}}$ that is the error of a closed loop control for the torque, where the reference torque is the output of a PI controller that implements a closed loop control for the $\omega_m$, this PI is represented in figure \ref*{fig: speed_controller}. The PI has an anti-windup strategy to avoid saturation's issues with related to the maximum torque.
% \begin{figure}[h!]
%     \centering
%     \includegraphics[width=0.75\textwidth]{speed_controller.png}
%     \caption{The speed controller}
%     \label{fig: speed_controller}
% \end{figure}
% \\
% Another important component is the VI-Estimator that given the the stator current and voltage estimates the stator flux and the torque.
% \begin{figure}[h!]
%     \centering
%     \includegraphics[width=1\textwidth]{VI_estimator.png}
%     \caption{The VI-Estimator}
%     \label{fig: VI_estimator}
% \end{figure}
% The conversion of the Space Vectors into 3-phase components and vice versa are made with the Clark transform and anti-transform.
% \begin{figure}
%     \centering
%     \begin{subfigure}[b]{0.4\textwidth}
%         \centering
%         \includegraphics[width=\textwidth]{clark_transform.png}
%         \caption{Clark transform}
%         \label{fig:clark_transform}
%     \end{subfigure}
%     \qquad
%     \begin{subfigure}[b]{0.4\textwidth}
%         \centering
%         \includegraphics[width=\textwidth]{anti_clark_transform.png}
%         \caption{Clark anti-transform}
%         \label{fig:anti_clark_transform}
%     \end{subfigure}
% \end{figure}



% % \begin{table}[htbp!]
% % \caption{Prima tabella}
% % \label{tab:prima_tabella}
% % \centering
% % \begin{tabular}{cccc}
% % \toprule
% % Prima Variabile	& Seconda Variabile	& Terza Variabile	& Quarta Variabile	\\
% % \midrule
% % XXX				& XXX				& XXX				& XXX				\\
% % \bottomrule	
% % \end{tabular}
% %\end{table}

% % \section{Terza Sezione}
% % \label{sec:terza_sezione_2}