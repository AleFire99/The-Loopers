\chapter{Problem Description}
\label{cha:problem_description}

    This report will describe the model of the system, our solution and some attempts to describe and control the system.

    The system is composed of a DC motor that provides torque to a metal base, over which a metal arm is fixed with an hinge and two symmetrical springs.

    The length of the arm, hence its inertia, and the the equilibiurm length of the springs can be modified in a variety of different configurations.

    \image{./images/Chapter 1/system.png}
    The system has several interfaces that could be connected to an acquisition system (DAC/ADC + Amplifier) to acquire measurements and provide input signal, namely:
    \begin{itemize}
        \item Actuators:
            \subitem Voltage driven DC motor;
        \item Sensors:
            \subitem Incremental Encoder for the position of the base \wrt the global reference frame;
            \subitem Incremental Encoder for the relative position of the arm \wrt the base.
    \end{itemize}
    The acquisition system composed by ADC/DAC + Amplifier is already configured, it doesn't require our attention, for this reason it will not be discussed in this report.

    The main task of this project is to provide a basic control strategy for such system and to develope a more advanced control strategy to accomodate all possible configurations of the system in question.
    
    This goal is divided in sub-tasks to be achieved:
    \begin{enumerate}
        \item position control of the base, with a frequency based approach;
        \item position control of the arm tip, with a frequency based or a state space approach;
        \item manage uncertainties and control the position of the arm tip with the system in different configurations, with a state space approach or other advanced control techniques.
    \end{enumerate}





